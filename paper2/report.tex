\documentclass[sigconf]{acmart}

\usepackage{graphicx}
\usepackage{hyperref}
\usepackage{todonotes}

\usepackage{endfloat}
\renewcommand{\efloatseparator}{\mbox{}} % no new page between figures

\usepackage{booktabs} % For formal tables

\settopmatter{printacmref=false} % Removes citation information below abstract
\renewcommand\footnotetextcopyrightpermission[1]{} % removes footnote with conference information in first column
\pagestyle{plain} % removes running headers

\newcommand{\TODO}[1]{\todo[inline]{#1}}

\begin{document}
\title{BigData Applications in Social Media for Marketing}


\author{Lokesh Dubey}
\orcid{hid309}
\affiliation{%
     \institution{Indiana University}
     \streetaddress{3209 E 10th St}
     \city{Bloomington} 
     \state{Indiana} 
     \postcode{47408}
}
\email{ldubey@indiana.edu}

% The default list of authors is too long for headers}
\renewcommand{\shortauthors}{L. Dubey}


\begin{abstract}
It has been widely accepted fact and has been addressed in many studies that Social Media data is prominently growing and diversified way of communication which entails immense possibilities with the data that it generates. It is not only a medium of socializing but a major platform for businesses to capitalize on this medium to increase their revenue in every way possible. Many new avenues of complexities and diversities introduced in social media data are identified and their respective challenges are stated. Challenges are identified in handling the social media data to derive correlations and implicit key performance indices to improve the efficacy and outreach of marketing campaigns of organizations. Big Data solutions to keep up with the paradigm shift in social media data and to remain focused towards upcoming possibilities are stated with some illustrations on political marketing and Internet Of Things.
\end{abstract}

\keywords{i523, hid309, Apache Spark, Hadoop, Social Media Analytics, Marketing}


\maketitle



\section{Introduction}

Social Media has become an integral part of any business working on larger scale and it has to be addressed without fail. With the advancements in technology and internet as a means of communication has drastically changed the way people interact today \cite{nickhajli}. Merely, the scale of internet and its availability has given a lot of business owners an opportunity to not rely on meetings in person for carrying out the business. The advancements, popularity of Social Media and equally thriving industry to provide much more advanced devices to have access to internet and social media has made it a major means of communication. Most prominently social media has become a means of entertainment as well \cite{clee}. Social Media is a platform where anyone at anytime can engage with a plethora of information. The type of information shared is just about everything. It is not confined by the limitations of television or film industry where a certain body, media broadcast companies, drive what content would be shared with the audience. Here any individual can just share any kind of information which has become a phenomenon where rest of the world can participate and exponential share that information \cite{clee}. And, just like any entertainment platform when there is an engagement of a very large amount of people it becomes a very viable and extremely profitable means for marketing and advertising.

Marketing on social media is, if not completely, a very different paradigm than any other kind of marketing. There is a multitude of avenues and areas to explore in social media that are to be explored in great detail to find the right strategy. It's not that its very difficult to advertise anything on social media, in fact, social media provides so much diversity in the ways to communicate with the customers that if used intelligently organizations can benefit a great deal from it and should be able to run their marketing campaigns much more efficiently than wasting a lot of resources on areas which are wasteful \cite{holly}. The reason behind this diversity is because the sheer amount detailed information that is available on social media. Unlike other platforms which have a high social presence or are largely a vital target area for marketing the information about the consumer remains either restricted or none. From other platforms it is mostly a prediction to identify the demographics of the individual engaging on that platform. Firstly, its not just the demographics which are useful to identify an organization's correct audience \cite{alexendras}. It has to be much more than that, the psychographics and general attitude of the consumers is equally important to identify and minimize the subset of audience to which the advertising should be directed. Various marketing campaigns can benefit a lot from this information because it reduces the cost of marketing while increasing the rates of conversion because of an intelligent choice of audience. The question arises here is how to get this information. And social media is all about data. In every minute on social media there are 500 hours of video uploaded on Youtube, 3.3 million Facebook posts are made, 3.8 Google Searches are done, around 400,000 tweets are posted on Twitter \cite{roberta}. This is what makes this a very interesting scenario, on one hand one gets a substantial amount of data daily which is a challenge to handle with any contemporary technology and on the other hand all of this data has nearly all the information that a marketing campaign would need to accurately target their consumer base. And this is where Big Data applications and contemporary technologies provide an equally interesting means of solving this problem.

\section{New avenues in Social Media Data}

As we established before, social media data offers immense possibilities of how individuals can handle the data. On various studies it has been already identified and is accepted by many organizations that the data to be tackled here is huge \cite{roberta}. However, just having a large amount of data is not the problem. It depends on how creatively the data can be utilized. Like other marketing mediums this is not all reliant only merely the number of views. It is much more than that and it entails practically endless amount of possibilities of what kind of data can be mined. This data can be useful on finding the demographics, psychographics, societal connections etc. In addition to that, this data can give marketing campaigns insights on things which were never even though useful before \cite{alexendras}. Organizations can try and delve into finding the whole lifecycle of the product being advertised right in front of them and can recursively keep improving on them. When a product is advertised for the first time one can get feedback and sentiments about the product right away from data itself and can tune, align their marketing strategy accordingly. On the other hand it is also a challenge that this is all happening publicly \cite{Horn2015}. And as much as this is helpful unlike any other one sided platforms where the end consumers of the product cannot engage on the platform, here the consumer can speak up and publicly announce the fallacies or even good reviews about the product.

In addition to these explicit data mining, there are many other levels on which data is can be mined for improving the performance of marketing campaigns \cite{michelen} \cite{mylynnfelt}. Not directly pertaining to the product at hand but all this data available on the consumers and their behaviors, social circle, likes, dislikes, can help with many other leads. They may provide a viable costumer, it can also be used to decide if the data from a certain individual is an outlier or indeed a reliable data. By looking at a behavior of a certain individual on social media and his/her interactions, information shared, social circle, likes and dislikes one can figure out of that individual is mischievous and highly unreliable or may be is simply a bot \cite{Ferrara}.

In addition to these avenues, and social media being biggest, richest and dynamic content generator, all of this we discussed is never constant. It keeps changing with new social media organizations trying to bring in more an more and features everyday and trying their best to engage more people on their sites. Which, in turn, increases the possibility of new type of data getting generated very often. One of the major paradigm shift in this direction has been Internet of Things (IOT) \cite{roberta2} \cite{neilp}. IOT is most inadvertently changing the completely dynamics of ever changing social media data. Organizations and other businesses are trying to capitalize their level best on this new technology which helps them introduce their products like washing machines, refrigerators, vending machines on the internet. As it provides excitement to their customers as this adds more level of comfort and trying new things has always been entertaining. For instance one organization tried to open up a vending machine based on a twitter hashtag \cite{neilp}. It's fun but this is an indication of on what level and how diverse this data can get in near future with IOT.

\section{Challenges of Social Media data and marketing}

Social media being biggest, richer and extremely dynamic makes it very big challenge for traditional technologies to tackle. However, merely the size of the data is not something new in it has been widely established the sheer amount of data in Social Media is huge and advancements, new technologies are an immediate requirement. Not only that, but these technologies needs to keep updating, refreshing themselves to keep up with the pace of new data that is being generated everyday and most importantly the type of data that keeps changing.

The most common challenges are more or less already acknowledged and there are solutions available for them or if not at least there are methodologies to handle those. Natural language processing, opinions, sentiment analysis are some common challenges which are handled with machine learning, many libraries which provide a certain level of accuracy in sentiment analysis. As far as the amount of data is concerned Big Data technologies like Hadoop MapReduce, Hive etc. are already being utilized to handle these scenarios. However, as stated before the social media space is extremely dynamic and organic space where the data is extremely volatile as far the type and life cycle of the data is concerned.  Besides, social media data in itself doesn't follow any pattern. Data and the information available can change substantially based on the type of social networking platform is under the scanner. For instance Instagram\footnote{"Instagram is a mobile, desktop, and Internet-based photo-sharing application and service that allows users to share pictures and videos either publicly or privately"} has very little to do with any textual information as its mostly about people posting pictures on it. There are certain ways to help image processing with the data available in form of comments and hashtags on the posts but its very limited. On the other hand sites like Facebook can be everything including, photos, textual data, audio files etc. The challenges are not only pertaining to the amount of data but how much any marketing campaigns wishes to gain from that data \cite{HOFACKER201673}. As there are immense possibilities of finding new insights and complex correlations which can in one way or other provided different key performance indices for increasing the efficacy of marketing campaign. And to help marketing campaigns and organizations the technical advancements and solutions are to be thought in greater detail to benefit marketing domain \cite{AMADO2017}.

\section{Big Data applications and capabilities}
The magnitude of social media data is not a news and there are many well defined architectures available and to an extent implemented as well which try to solve this problem in a most generic way \cite{Olshannikova2017}. We can agree on this partially that the amount of data and how that will be handled has been identified and there are technologies available which are fairly scalable to not pause any limitation in near future. However the bigger challenges lately have not been just the amount of data but the type of data and the methodologies which can be utilized, while not compromising on amount of data that can be handled, to get the most important and vital piece of information from the data. As stated before data in recent past is not merely about counts and views \cite{mylynnfelt}. Data contains a lot more of insights and with ever changing social media market and its acceptance it is almost always volatile.

With recently growth of technologies like Apache Spark\footnote{"Apache Spark is an open-source cluster-computing framework"} and its extremely diverse set of APIs it provides a great customizable platform which can be hosted on any of the Big Data solutions like Apache HDFS \footnote{"Apache Hadoop is an open-source software framework used for distributed storage and processing of dataset of big data using the MapReduce programming model."} for distributed computing to ensure that organizations are getting the best out of there data and are not lagging behind because of technological limitations. There are more details provided later for some sample solutions where Spark can be used but for now we can draw some parallels on what kind of useful performance indices can be retrieved from social media data and how Big Data solutions can be useful there. Apart from the size of the data and the reception of a particular post or an ad in any social site there are lot of works done around natural language processing for sentiment analysis of the data \cite{IJIRSTV1I11036}. However, these details mostly pertain only to opinion mining, sentiment analysis of how the posts or products were perceived by the audiences \cite{Batrinca2015}. But there's more to this data. Various libraries of Spark like Machine Learning Library\footnote{"MLlib is Apache Spark's scalable machine learning library."} and GraphX\footnote{"GraphX is Apache Spark's API for graphs and graph-parallel computation."} can be utilized to build models which are not merely dependent on sentiments or opinions but to mine data on how the product is being perceived by the consumers as a group or as a societal importance. Which can of course be related to the preexisting sentiment analysis by natural language processing but this provides an acute subset to work with as this can minimize the target audience to a certain extent.

Twitter\footnote{"Twitter is an online news and social networking service where users post and interact with messages called tweets"} tweets can be collected and related to each other with a GraphX library and models can be created to find out what causes anything to go on a trending page or what are the nitty gritties that can get consumers of twitter more excited about any product which of course increases the footprint of any advertisement more. This helps however only with increasing the outreach of the particular advertisement. Or on the other hand it can help create a model which can identify, by the interactions of different user with other users, to identify social media bots which are not a potential customer for the organizations which should be weeded out while considering and calculating any kind of the key performance indices by a marketing campaign.

\section{Political marketing}
As much as marketing of commercial products has its own space in social media, it has a very huge impact on political marketing as well. And there are two sides to this which were exacerbated and surfaced in recent US presidential elections. Marketing on social media for any political campaign has two challenges. Primarily it is to promote one particular individual like any other traditional marketing but on the flip side of it one has to make sure that there are no entities are utilizing the same platform to alter the results of the election by utilizing the openness of social media and promoting fake information which most of consumers can readily believe in without ensuring the authenticity of the posts. Which in traditional terms was also considered as echo chambers or filter bubbles where they are kept aloof from contrary perspectives and possibly reality. Post 2016 US presidential elections it was researched that a lot of fake Facebook posts were circulated from within the country and even from other countries \cite{NBERw23089} \cite{fbuselection}. This represents a new and a unique challenge for the marketing world as this is totally different type of data to deal with, compared to commercial marketing and also its a unique type of data mining.

In this particular scenario even for a giant organization like Facebook took around 6 to 7 months to identify the authenticity of the claim of fake news and to weed out and identify the exact ads and posts which were not authentic \cite{fbuselection2}. However, this can be done proactively by the political campaigns proactively to ensure that no unethical activity on social media can affect their campaigns.

Data science methodologies and a spark platform can provide a lot of insights on this particular field. Most of this data is generated and fake accelerated in form of making it looking like a trending topic with endorsements showing that many people like it and that it was shared by numerous people and it's done by social media bots \cite{Lilian2013srep}. Many of the bots can be utilized to create fake debates on fake posts which can also be perceived by the audience as firstly true and secondly being endorsed by a lot of other social media users which can help tilting or at least affecting the decision of a voter \cite{andrejz}.

To avoid this with all the data available on social media is crucial and helpful. All the contemporary key indices which are retrieved from the data today in form of sentiment analysis, opinion mining, counts of reposts, share can play a very vital role in this kind of data mining but to a larger extent this information also needs to be supported by behavioral information which can be retrieved from the data by building correlation between the various posts and data that is shared by the social media users. Spark and its library can be used to build graph based models to generate and weed out the social media handles/accounts which seem to be posting unethical information and will help to weed out those kinds of accounts to avoid swaying away any voter's decision in wrong direction.

\section{Internet Of Things}
Imagine one individual's washing machine posting a Tweet on manufacturer's page publicly about a certain anomaly. Or may be posting stats of its last week run and how much water it saved or how much electricity it wasted. Technology world is on the verge of two possibly biggest data generators merging together. Devices at a consumers home may remain constantly connected with its owners even while he's out of town. They might be able to connect to other devices in the vicinity and possibly identify if the issue it is facing to function is isolated or a general problem in the community. Essentially, we are looking at a dawn of endless possibilities as initially was thought about social media.

There are many aspects to IOT which can exponential increase the complexity and richness of the data available on social media. On one hand this may also, as it's usually the case with any technology, pause a threat on privacy etc. but this will continue to thrive on the basis of providing at most comfort level to the consumer of devices \cite{iotlawrence}. The variations in the richness of data and complexity would be endless. There could be patients and their devices altering their doctors directly on their social media handles. The information can be broadcasted on multiple groups probably for the hospital employees to ensure their patients are constantly monitored even when its not physically possible. Every now and then certain group of devices like refrigerators or microwave ovens of a certain brand may start showing up on a trending page because they have been saving so much enery lately and they have received rave reviews about them on social media by their owners.

Yet again, we're talking about the endless amount of rich and diverse data which opens up the doors for numerous of ways in which this data can be utilized. With IOT organizations can utilize the data available from the devices and their posts on social media to garner statistics on how well their devices are working, how good is a reception of the devices, in fact, this might even be recursive. In every which way, the quest for enriching lives of humans would continue and by learning from the data posted by the devices and their consumers more avenues of possibilities will keep opening to make it better and better each day.

Big Data technologies very soon will be utilized to make this possible. Today, IOT is mostly focused towards data that is posted directly from the devices to certain servers of the manufacturer or the maintenance and it is indeed large enough to be requiring Big Data technologies to handle it. However, with social media this will give another good use case to make predictions. Let's say based on the social impressions of the owners of washing machines complaining a lot about their machine's not working in the optimal way lately to their friends and many washing machines of that particular area possibly posting an collaborating on certain groups on social sites about water being to hard lately may help identify the issues and identify in exact which locality this is happening and will help rectify the issue permanently as soon as it starts to appear. This exercise can again be fed back the system and marketed as level of comfort and proactive rectification their devices provide to the customer in form of more social media impressions which may in turn also increase there consumer base.

\section{Conclusion}
Various different proliferating avenues in social media that are changing the diversity, richness and complexity of the data available on social media that can help marketing campaigns and various organizations in many different ways. This data is so enriched and perplexing that marketing industry can choose based on their criteria on how much they wish to delve into this data. More information is never bad but handling that data and being able to mine it for different key performance indices can be made possible with the contemporary Big Data technologies. Growing and organically changing social media data might have made it difficult for various traditional approaches of identifying the right strategies of marketing but with latest Big Data technologies like Spark Streaming, MLib, GraphX it is possible to tackle any kind of data in the most fluid form so that it is customizable enough to quickly adapt based on the changing needs of social media data. There are many challenges on handling this new diversity and complexity of this data in form of more detailed information that is available other than simple number of views or demographics. It is established that the availability of data and insights to find complex correlations between the data is practically endless and it relies highly on the will and to the extent marketing campaigns wish to increase their efficacy of their marketing which makes this more complex and requires more creativity. Almost all the type of data that is being generated and is available has been well thought of and aligns well with technology available in Big Data. However, it has been done mostly in social media and Big Data spaces individually. There's a high need to focus on more solutions which are more targeted and focus towards the major paradigm shift and viral behavior of social media. Which is also at the brink of increasing and diversifying immensely after various different new usages of social media like its use in politics and Internet Of Things.

\begin{acks}
The author would like to thank Dr. Gregor von Laszewski for his support and suggestions for this review.
\end{acks}

\bibliographystyle{ACM-Reference-Format}
\bibliography{report} 


\end{document}
